%Udarbejdet af Magnus Overby
% Her definerer jeg min "makro" som laver en udregning for mig
\newenvironment{spændingsfaldsum}[3]
%{ dU1 | dU2 | dU3 | dU4 | dU5 }{dU-resultat navn}{dU1 navn | dU2 navn | dU3 navn | dU4 navn | dU5 navn}
{
%bestemmer seperator samt læser lister
\setsepchar{ | }
\readlist\DeltaU{#1}
\readlist\Nam{#3}
%Her udskiftes komma med punktum
\StrSubstitute{\DeltaU[1]}{,}{.}[\et]%
\StrSubstitute{\DeltaU[2]}{,}{.}[\to]%
\StrSubstitute{\DeltaU[3]}{,}{.}[\tre]%
\StrSubstitute{\DeltaU[4]}{,}{.}[\fire]%
\StrSubstitute{\DeltaU[5]}{,}{.}[\fem]%
%Da kvadratrod 3 bruges ofte defineres den her.
\FPeval\sqTre{root(2,3)}

\FPeval\tot{round(100*(\et + \to + \tre + \fire + \fem):2)}

%tjekker om der er parrellel kabler og ændre til output

%tests "flag" til hvor mange impedanser der er.
\FPeval\yepET{1}
\FPeval\yepTO{1}
\FPeval\yepTRE{1}
\FPeval\yepFI{1}
\FPeval\yepFE{1}

\FPeval\antaltestet{\et}
\FPifzero{\antaltestet}
\begin{centering}
	Ingen impedans, prøv igen\\
\end{centering}
\else
\FPeval\yepET{0}
\fi

\FPeval\antaltestto{\to + \yepET}
\FPifzero{\antaltestto}
\begin{centering}
	En impedans... hvad regnede du med?\\
\end{centering}
\else
\FPeval\yepTO{0}
\fi

\FPeval\antaltesttre{\tre + \yepET + \yepTO}
\FPifzero{\antaltesttre}
\begin{align*}
	U_{#2} 	&= \Delta U_{1} + \Delta U_{2}\\
	&=  \et + \to \\
	&= \tot \%
\end{align*}
\else
\FPeval\yepTRE{0}
\fi


\FPeval\antaltestfire{\fire + \yepET + \yepTO + \yepTRE}
\FPifzero{\antaltestfire}
\begin{align*}
	\Delta U_{#2} 	&= \Delta U_{\Nam[1]} + \Delta U_{\Nam[2]} + \Delta U_{\Nam[3]} \\
	&= \et + \to + \tre\\
	&=\tot \%
\end{align*}
\else
\FPeval\yepFI{0}
\fi



\FPeval\all{\fem + \yepET + \yepTO + \yepTRE + \yepFI}
\FPifzero{\all}
\begin{align*}
	U_{#2} 	&= \Delta U_{1} + \Delta U_{2} + \Delta U_{3} + \Delta U_{4} \\
			&= \et + \to + \tre + \fire \\
			&=\tot \%
\end{align*}
\else
\FPeval\yepFE{0}
\fi




\FPeval\all{\yepET + \yepTO + \yepTRE + \yepFI + \yepFE}
\FPifzero{\all}
	\begin{align*}
		U_{#2} 	&= \Delta U_{1} + \Delta U_{2} + \Delta U_{3} + \Delta U_{4} + \Delta U_{5}\\
				&= \et + \to + \tre + \fire + \fem \\
				&= \tot \%
	\end{align*}
\else
\fi
}


