%Udarbejdet af Jakob Dupont

% Her definerer jeg min "makro" som laver en udregning for mig
\newenvironment{FraIkTilZ}[5]%{Ik}{R/X}{Un}{Z navn}{Ik navn}
{
% I følgende udskifter jeg kommaer med punktum i mine variabler. FP-pakken regner med punktum.
% Hver variabel starter med # og derefter nummeret i rækken.
% Fx er SK værdien variabel nummer 3
\StrSubstitute{#1}{,}{.}[\Ik]%
\StrSubstitute{#2}{,}{.}[\RX]%
\StrSubstitute{#3}{,}{.}[\Un]%
%Da kvadratrod 3 bruges ofte defineres den her.
\FPeval\sqTre{root(2,3)}

% Resultat i radianer
\FParccot\InverseCot{\RX}
%Først skal vi udregne R
\FPcos\cosPhi{\InverseCot}
\FPeval\R{\Un / \Ik * \cosPhi}
%Udregn X
\FPsin\sinPhi{\InverseCot}
\FPeval\X{\Un / \Ik * \sinPhi}

\def\enhed{}%

% Her laver jeg en if/else hvor jeg kigger på spændingen. Er spændingen over 500 V regner jeg i ohm - ellers milliohm. På den måde kan jeg bruge det i både LV og HV
\FPifgt{\Un}{500}%
\renewcommand\enhed{\ohm}%
\FPeval\Rrounded{clip(round(\R:2))}
\FPeval\Xrounded{clip(round(\X:2))}
\else
 %Lav til milli
\FPeval\R{clip(\R * 1000)}
\FPeval\Rrounded{clip(round(\R:2))}
\FPeval\X{clip(\X * 1000)}
\FPeval\Xrounded{clip(round(\X:2))}
\renewcommand\enhed{\milli \ohm}%
\fi

\begin{equation*}
	\begin{split}
		Z_{#4} 	&= \frac{U_{n}}{I_{k,#5}} \angle cot^{-1}(R/X)\\
				&= \frac{\Un}{\Ik} \angle cot^{-1}(\RX)\\
				&= \Rrounded + j\Xrounded \unit[]{\enhed}
	\end{split}
\end{equation*}
}


