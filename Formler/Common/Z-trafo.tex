%Udarbejdet af Magnus Overby

% Her definerer jeg min "makro" som laver en udregning for mig
\newenvironment{Ztrafo}[5]%{Un}{Sn}{ek}{Pcu}{impedans navn}
{
% I følgende udskifter jeg kommaer med punktum i mine variabler. FP-pakken regner med punktum.
% Hver variabel starter med # og derefter nummeret i rækken. Og så hvilket navn de får videre hen.
\StrSubstitute{#1}{,}{.}[\Un]%
\StrSubstitute{#2}{,}{.}[\Sn]%
\StrSubstitute{#3}{,}{.}[\ek]%
\StrSubstitute{#4}{,}{.}[\Pcu]%

%Først regnes størelsne af Z
\FPeval\Z{(\Un^2 * \ek)/\Sn}

%Udregn vinklen i radianer...
\FPeval\Cos{\Pcu/(\Sn * \ek)}
\FParccos\PF{\Cos}

% R værdi
\FPeval\Rrounded{clip(round(\Z * \Cos:2))}

% X værdi
\FPsin\Sin{\PF}
\FPeval\Xrounded{clip(round(\Z * \Sin:2))}

\def\enhed{}%
% Her laver jeg en if/else hvor jeg kigger på spændingen. Er spændingen over 500 V regner jeg i ohm - ellers milliohm. På den måde kan jeg bruge det i både LV og HV
\FPifgt{\Un}{500}%
\renewcommand\enhed{\ohm}%
\else
 %Lav til milli
\FPeval\R{\Z * \Cos}
\FPeval\Rrounded{clip(round(\R * 1000:2))}
\FPeval\X{\Z * \Sin}
\FPeval\Xrounded{clip(round(\X * 1000:2))}
\renewcommand\enhed{\milli \ohm}%
\fi


\begin{align*}
Z_{trafo,#5} 	&= \frac{U_{n}^{2} \cdot e_{k}}{S_{n}} \angle cos^{-1}\left(\frac{P_{cu}}{S_{n} \cdot e_{k}}\right) \\
			&= \frac{\Un^{2} \cdot \ek}{\Sn} \angle cos^{-1}\left(\frac{\Pcu}{\Sn \cdot \ek}\right) \\
			&= \Rrounded + j\Xrounded \unit{\enhed}
\end{align*}
}


