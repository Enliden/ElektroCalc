%Udarbejdet af Magnus Overby
% Her definerer jeg min "makro" som laver en udregning for mig
\newenvironment{HV-lækstrøm-pr-fase}[5]
%{Un}{længde af kabler}{Kapitans pr km}{frekvens}{kabel navn | Z1 navn | Z2 navn | Z3 navn}
{
\setsepchar{ | }
\readlist\Nam{#5}
%Her udskiftes komma med punktum
\StrSubstitute{#1}{,}{.}[\Un]%
\StrSubstitute{#2}{,}{.}[\Lk]%
\StrSubstitute{#3}{,}{.}[\Ck]%
\StrSubstitute{#4}{,}{.}[\fr]%
%Da kvadratrod 3 bruges ofte defineres den her.
\FPeval\sqTre{root(2,3)}
%først udregnes den samlede kapacitans
\FPeval\Co{round(\Lk * \Ck:2)}
%Så regnes den kapacitive reaktans for kablet
\FPeval\Xk{round(1/((2 * pi * \fr * \Co)/1000000):2)}
%Udregn så lækstrømmen
\FPeval\ICo{round(\Un / (\sqTre * \Xk):2)}
%
\begin{align*}
	C_0 &= l_{\Nam[1]} \cdot c_0\\
		&= \Lk \cdot \Ck\\
		&= \Co \mu f\\
	X_C &= \frac{1}{2\pi \cdot f \cdot C_0 \cdot 10^{-6}}\\
		&= \frac{1}{2\pi \cdot \fr \cdot \Co \cdot 10^{-6}}\\
		&= \Xk \Omega\\
	I_{C0}	&= \frac{U_n}{\sqrt[]{3} \cdot X_C}\\
			&= \frac{\Un}{\sqrt[]{3} \cdot \Xk}\\
			&=\ICo A
\end{align*}
}



