%Udarbejdet af Magnus Overby
% Her definerer jeg min "makro" som laver en udregning for mig
\newenvironment{LV-Ztotal-max}[4]
%{ R1 | X1 | R2 | X2 | R3 | X3 | R4 | X4 | R5 | X5 }{Z resultat navn}{Z1 navn | Z2 navn | Z3 navn | Z4 navn | Z5 navn}{Antal_kab 1 | Antal kabler 2 | Antal kabler 3 | Antal kabler 4 | Antal kabler 5}
{
%bestemmer seperator samt læser lister
\setsepchar{ | }
\readlist\Zv{#1}
\readlist\Nam{#3}
\readlist\nKab{#4}
%Her udskiftes komma med punktum
\StrSubstitute{\Zv[1]}{,}{.}[\Ret]%
\StrSubstitute{\Zv[2]}{,}{.}[\Xet]%
\StrSubstitute{\Zv[3]}{,}{.}[\Rto]%
\StrSubstitute{\Zv[4]}{,}{.}[\Xto]%
\StrSubstitute{\Zv[5]}{,}{.}[\Rtre]%
\StrSubstitute{\Zv[6]}{,}{.}[\Xtre]%
\StrSubstitute{\Zv[7]}{,}{.}[\Rfi]%
\StrSubstitute{\Zv[8]}{,}{.}[\Xfi]%
\StrSubstitute{\Zv[9]}{,}{.}[\Rfe]%
\StrSubstitute{\Zv[10]}{,}{.}[\Xfe]%
\StrSubstitute{\nKab[1]}{,}{.}[\nEt]%
\StrSubstitute{\nKab[2]}{,}{.}[\nTo]%
\StrSubstitute{\nKab[3]}{,}{.}[\nTre]%
\StrSubstitute{\nKab[4]}{,}{.}[\nFi]%
\StrSubstitute{\nKab[5]}{,}{.}[\nFe]%
%Da kvadratrod 3 bruges ofte defineres den her.
\FPeval\sqTre{root(2,3)}
	
%Udregn samlede R og X (ganger 1,8 med R for at få max værdi)
\FPeval\Rtot{round(1.8 * (\Ret/\nEt + \Rto/\nTo + \Rtre/\nTre + \Rfi/\nFi + \Rfe/\nFe):2)}
\FPeval\Xtot{round(\Xet/\nEt + \Xto/\nTo + \Xtre/\nTre + \Xfi/\nFi + \Xfe/\nFe:2)}
	
%udregn samlede impedans
\FPeval\Ztot{root(2,pow(2,\Rtot)+pow(2,\Xtot))}

%tjekker om der er parrellel kabler og ændre til output
\FPifgt\nEt{1}
\newcommand\Zet{\frac{Z_{\Nam[1]}}{n_{\Nam[1]}}}
\newcommand\RXet{\frac{(1,8 \cdot \Ret + j \Xet)}{\nEt}}
\else
\newcommand\Zet{Z_{\Nam[1]}}
\newcommand\RXet{(1,8 \cdot \Ret + j \Xet)}
\fi

\FPifgt\nTo{1}
\newcommand\Zto{\frac{Z_{\Nam[2]}}{n_{\Nam[2]}}}
\newcommand\RXto{\frac{(1,8 \cdot \Rto + j \Xto)}{\nTo}}
\else
\newcommand\Zto{Z_{\Nam[2]}}
\newcommand\RXto{(1,8 \cdot \Rto + j \Xto)}
\fi

\FPifgt\nTre{1}
\newcommand\Ztre{\frac{Z_{\Nam[3]}}{n_{\Nam[3]}}}
\newcommand\RXtre{\frac{(1,8 \cdot \Rtre + j \Xtre)}{\nTre}}
\else
\newcommand\Ztre{Z_{\Nam[3]}}
\newcommand\RXtre{(1,8 \cdot \Rtre + j \Xtre)}
\fi

\FPifgt\nFi{1}
\newcommand\Zfi{\frac{Z_{\Nam[4]}}{n_{\Nam[4]}}}
\newcommand\RXfi{\frac{(1,8 \cdot \Rfi + j \Xfi)}{\nFi}}
\else
\newcommand\Zfi{Z_{\Nam[4]}}
\newcommand\RXfi{(1,8 \cdot \Rfi + j \Xfi)}
\fi

\FPifgt\nFe{1}
\newcommand\Zfe{\frac{Z_{\Nam[5]}}{n_{\Nam[5]}}}
\newcommand\RXfe{\frac{(1,8 \cdot \Rfe + j \Xfe)}{\nFe}}
\else
\newcommand\Zfe{Z_{\Nam[5]}}
\newcommand\RXfe{(1,8 \cdot \Rfe + j \Xfe)}
\fi

%tests "flag" til hvor mange impedanser der er.
\FPeval\yepET{1}
\FPeval\yepTO{1}
\FPeval\yepTRE{1}
\FPeval\yepFI{1}
\FPeval\yepFE{1}

\FPeval\Zettest{\Ret + \Xet}
\FPifzero{\Zettest}
\begin{centering}
	Ingen impedans, prøv igen\\
\end{centering}
\else
\FPeval\yepET{0}
\fi

\FPeval\Ztotest{\Rto + \Xto + \yepET}
\FPifzero{\Ztotest}
\begin{centering}
	En impedans... hvad regnede du med?\\
\end{centering}
\else
\FPeval\yepTO{0}
\fi

\FPeval\Ztretest{\Rtre + \Xtre + \yepET + \yepTO}
\FPifzero{\Ztretest}
\begin{align*}
Z_{#2,max} 	&= \Zet + \Zto\\
		&= \RXet + \RXto\\
		&=\Rtot + j\Xtot m\Omega
\end{align*}
\else
\FPeval\yepTRE{0}
\fi

\FPeval\Ztretest{\Rfi + \Xfi + \yepET + \yepTO + \yepTRE}
\FPifzero{\Ztretest}
\begin{align*}
Z_{#2,max} 	&= \Zet + \Zto + \Ztre\\
		&= \RXet + \RXto + \RXtre\\
		&=\Rtot + j\Xtot m\Omega
\end{align*}
\else
\FPeval\yepFI{0}
\fi

\FPeval\Ztretest{\Rfe + \Xfe + \yepET + \yepTO + \yepTRE + \yepFI}
\FPifzero{\Ztretest}
\begin{align*}
Z_{#2,max} 	&= \Zet + \Zto + \Ztre + \Zfi\\
		&= \RXet + \RXto + \RXtre + \RXfi\\
		&=\Rtot + j\Xtot m\Omega
\end{align*}
\else
\FPeval\yepFE{0}
\fi

\FPeval\all{\yepET + \yepTO + \yepTRE + \yepFI + \yepFE}
\FPifzero{\all}
\begin{align*}
Z_{#2,max} 	&= \Zet + \Zto + \Ztre + \Zfi + \Zfe\\
		&= \RXet + \RXto + \RXtre + \RXfi + \RXfe\\
		&=\Rtot + j\Xtot m\Omega
\end{align*}
\else
\fi
}


