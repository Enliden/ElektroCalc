%Udarbejdet af Magnus Overby
% Her definerer jeg min "makro" som laver en udregning for mig
\newenvironment{KB-kontrol-tid}[6]
%{aflæst energi}{eksponent}{k-værdi}{tværsnit}{I^2t navn | k navn | tværsnit navn}
{
	\setsepchar{ | }
	\readlist\Nam{#6}
	%Her udskiftes komma med punktum
	\StrSubstitute{#1}{,}{.}[\strøm]%
	\StrSubstitute{#2}{,}{.}[\eksTal]%
	\StrSubstitute{#3}{,}{.}[\tid]%
	\StrSubstitute{#4}{,}{.}[\K]%
	\StrSubstitute{#5}{,}{.}[\S]%
	%Da kvadratrod 3 bruges ofte defineres den her.
	\FPeval\sqTre{root(2,3)}
	
	%Udregn eksponenten til tal

	%Udregn tilladelig gennemslip. Sørger for de har samme eksponent
	\FPeval\tilGen{pow(2,\K * \S)}
	\FPeval\TGSL{round(\tilGen / pow(\eksTal,10) :0)}
	
	\FPeval\Energigennemslip{ round((\strøm)^2 * \tid:2) }
	\FPeval\EGSL{round(\Energigennemslip / pow(\eksTal,10) :0)}
	
	\def\truefalse{}
	
	\FPifgt{\tilGen}{\Energigennemslip} 
	\renewcommand\truefalse{OK}
	\else
	\renewcommand\truefalse{Ikke \hspace{4pt} OK}
	\fi
	
	\begin{align*}
		I_{\Nam[1]}^2 \cdot t &\le k^2 \cdot S^2 \\
	%
		\strøm^2 \cdot \tid	&\le \K ^2 \cdot \S ^2 \\
	%
		\EGSL \cdot 10^{\eksTal}	&\le \TGSL \cdot 10^{\eksTal}   \longrightarrow  \truefalse
	\end{align*}
}



