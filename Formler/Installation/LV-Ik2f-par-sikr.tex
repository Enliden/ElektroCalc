%Udarbejdet af Magnus Overby
% Her definerer jeg min "makro" som laver en udregning for mig
\newenvironment{LV-Ik2f,parSikr-kA}[3]
%{ R net + trafo | X net + trafo | R kabler | X kabler }{antal stikledninger}{Ik,navn | net-trafo-navn | kabel-navn}
%impedanser i milliohm
{
\setsepchar{ | }
\readlist\Zv{#1}
\readlist\Nam{#3}
%Her udskiftes komma med punktum
\StrSubstitute{\Zv[1]}{,}{.}[\Rnt]%
\StrSubstitute{\Zv[2]}{,}{.}[\Xnt]%
\StrSubstitute{\Zv[3]}{,}{.}[\Rk]%
\StrSubstitute{\Zv[4]}{,}{.}[\Xk]%
\StrSubstitute{#2}{,}{.}[\nKab]%
%Da kvadratrod 3 bruges ofte defineres den her.
\FPeval\sqTre{root(2,3)}

%Udregn R og X
\FPeval\Rtot{ 2 *(\Rnt + \Rk * (1/(\nKab - 1) + 1))}
\FPeval\Xtot{ 2 *(\Xnt + \Xk * (1/(\nKab - 1) + 1))}

%udregn samlede net og trafo impedans
\FPeval\Ztot{root(2,pow(2,\Rtot)+pow(2,\Xtot))}

%udregn kortslutningen 1f
\FPeval\Ikf{round(400 / (\Ztot):2)}

\begin{align*}
Ik_{\Nam[1]}	&= \frac{U_n}{2 \cdot \left(Z_{\Nam[2]} + \left(\dfrac{Z_{\Nam[3]}}{n_{\Nam[3]} - 1} + Z_{\Nam[3]}\right)\right)}\\
				&= \frac{400}{2 \cdot \left((\Rnt + j\Xnt) + \left(\dfrac{(\Rk + j\Xk)}{\nKab - 1} + (\Rk + j\Xk)\right)\right)}\\
				&= \Ikf kA
\end{align*}
}


