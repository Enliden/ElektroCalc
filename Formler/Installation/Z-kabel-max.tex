%Udarbejdet af Magnus Overby

% Her definerer jeg min "makro" som laver en udregning for mig
\newenvironment{Zkabel-max}[6]%{L(km)}{r}{x}{Un}{KabelNavn}{1,8/1,5}
{
% I følgende udskifter jeg kommaer med punktum i mine variabler. FP-pakken regner med punktum.
% Hver variabel starter med # og derefter nummeret i rækken. Og så hvilket navn de får videre hen.
\StrSubstitute{#1}{,}{.}[\l]%
\StrSubstitute{#2}{,}{.}[\r]%
\StrSubstitute{#3}{,}{.}[\x]%
\StrSubstitute{#4}{,}{.}[\Un]%
\StrSubstitute{#6}{,}{.}[\K]%

%R værdi
\FPeval\R{\K * \l * \r}

%X værdi
\FPeval\X{\l * \x}


\def\enhed{}%
% Her laver jeg en if/else hvor jeg kigger på spændingen. Er spændingen over 500 V regner jeg i ohm - ellers milliohm. På den måde kan jeg bruge det i både LV og HV
\FPifgt{\Un}{500}%
\FPeval\Rrounded{clip(round(\R * 1:2))}
\FPeval\Xrounded{clip(round(\X * 1:2))}
\renewcommand\enhed{\ohm}%
\else
 %Lav til milli
\FPeval\Rrounded{clip(round(\R * 1000:2))}
\FPeval\Xrounded{clip(round(\X * 1000:2))}
\renewcommand\enhed{\milli \ohm}%
\fi


\begin{align*}
Z_{kabel,#5} = l \cdot (\K \cdot r + jx) = \l \cdot (\K \cdot \r + j\x) = \Rrounded + j\Xrounded \unit{\enhed}
\end{align*}
}


