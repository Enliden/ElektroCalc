%Udarbejdet af Magnus Overby

% Her definerer jeg min "makro" som laver en udregning for mig
\newenvironment{faseKOMP-Iny}[5]
%{I_belastning}{cosphi_før}{cosphi_ønsket}{Q_batteristørrelse}{Tavlenavn}
{
%Her udskiftes komma med punktum
\StrSubstitute{#1}{,}{.}[\Ib]%
\StrSubstitute{#2}{,}{.}[\cphif]%
\StrSubstitute{#3}{,}{.}[\cphief]%
\StrSubstitute{#4}{,}{.}[\Qbat]%
\readlist\arg{#5}
%Da kvadratrod 3 bruges ofte defineres den her.
\FPeval\sqTre{root(2,3)}

%Udregn effekten for belastningen
\FPeval\Pfor{400 * \Ib * \sqTre * \cphif}
\FPeval\PkW{round(\Pfor/1000:2)}

%Udregner den øsnkede reaktive effekt fra batterier
\FParccos\phif{\cphif}
\FParccos\phief{\cphief}
\FPtan\tanPhif{\phif}
\FPtan\tanPhief{\phief}
\FPeval\QbatOE{\Pfor * (\tanPhif - \tanPhief)}
\FPeval\QbatOEk{round(\QbatOE / 1000:2)}

%Antal ønskede batterier
\FPeval\AnBat{round(\QbatOEk / \Qbat:2)}
\FPeval\AnBatUP{round(trunc(\AnBat:0)+0.5:0)}


%Det der skrives i opgaven%
Først regnes det samlede effektforbrug.
$$ P_{A1} = U_{n} \cdot I_{\arg[1]} \cdot \sqrt{3} \cdot cos \varphi_{før} = 400 \cdot \Ib \cdot \sqrt{3} \cdot \cphif = \PkW \unit{kW}$$
Vi kan så regnede den ønskede reaktiv effekt fra batterierne. Vi ønsker en $cos \varphi$ på \cphief.

\begin{align*}
Q_{ønsket}  &= P_{\arg[1]} \cdot (tan \varphi_{før} - tan \varphi_{efter})\\
			&= \PkW \cdot (tan(cos^{-1}(\cphif)) - tan(cos^{-1}(cphief)))\\
			&= \QbatOEk \unit{kVAR}\\
%
\intertext{Vi kan så udregne hvor mange batterier vi skal bruge af størrelsen \Qbat \unit{kVAR}.}
%
Antal_{batterier} &= \frac{Q_{ønsket}}{Q_{batteri}}\\
				&= \frac{\QbatOEk}{\Qbat}\\
				&= \AnBat \approx \AnBatUP\\
\end{align*}

Vi runder op for at komme tætter på $cos \varphi = 1$. Vi kan nu udregne den nye belastningsstrøm.

\begin{align*}\\
Q_{før} &= U_{n} \cdot I_{A1} \cdot \sqrt{3} \cdot sin \varphi\\
		&= 400 \cdot 526.41 \cdot \sqrt{3} \cdot sin(cos^{-1}(0.82))\\
		&= 209kVAR\\
Q_{rest} &= Q_{før} - Q_{totBatteri}\\
		&= 209 - (10 \cdot 7)\\
		&= 139kVAR\\
\varphi_{ny}		&= tan^{-1}\left(\frac{Q_{rest}}{P_{A1}}\right)\\
					&= 24.9 \degree\\
cos \varphi_{ny} &= cos(24.9) = 0,91\\
I_{ny,A1} &= \frac{P_{A1}}{U_{n} \cdot \sqrt{3} \cdot cos \varphi_{ny}}\\
			&= \frac{299000}{400 \cdot \sqrt{3} \cdot 0.91}\\
			&= 474A
\end{align*}

Strømmen efter fasekompenseringer bliver altså $474 A$ $cos \varphi = 0.91$
}


