\documentclass[11pt]{report}


%\usepackage[danish]{babel}%https://ctan.org/pkg/babel (Multilingual support for Plain TEX or LATEX)
\usepackage[utf8]{inputenc}%https://ctan.org/pkg/inputenc (Accept different input encodings)
\usepackage{amsmath}%https://ctan.org/pkg/amsmath (AMS mathematical facilities for LATEX)
\usepackage{mathtools}%https://ctan.org/pkg/mathtools (Mathematical tools to use with amsmath)
\usepackage{icomma}%https://ctan.org/pkg/icomma (Intelligent commas for decimal numbers)
\usepackage{siunitx}%https://ctan.org/pkg/siunitx (A comprehensive (SI) units package)
\usepackage{fp}% https://ctan.org/pkg/fp?lang=en (Fixed point arithmetic)
\usepackage{xstring}%https://ctan.org/pkg/xstring?lang=en (String manipulation for (LA)TEX)
\usepackage{xstring}%https://ctan.org/pkg/xstring?lang=en (String manipulation for (LA)TEX)
\usepackage{amsmath}%https://ctan.org/pkg/amsmath (AMS mathematical facilities for LATEX)
\usepackage{mathtools}%https://ctan.org/pkg/mathtools (Mathematical tools to use with amsmath)
\usepackage{calculator}%https://ctan.org/pkg/calculator (Use LATEX as a scientific calculator)
\usepackage{icomma}%https://ctan.org/pkg/icomma (Intelligent commas for decimal numbers)
\usepackage{siunitx}%https://ctan.org/pkg/siunitx (A comprehensive (SI) units package)
%\usepackage{tabu}%https://ctan.org/pkg/tabu (Flexible LATEX tabulars)
\usepackage{ifthen}%https://ctan.org/pkg/ifthen (Conditional commands in LATEX documents)
\usepackage{fp}% https://ctan.org/pkg/fp?lang=en (Fixed point arithmetic)
\usepackage{xintexpr}%https://ctan.org/tex-archive/macros/generic/xint ()
\usepackage{etoolbox}%https://ctan.org/pkg/etoolbox (e-TEX tools for LATEX)
\usepackage{makecell}%https://ctan.org/pkg/makecell (Tabular column heads and multilined cells)
\usepackage{bm}%https://ctan.org/pkg/bm (Access bold symbols in maths mode)
\usepackage{xfp}



% Jeg bruger myspace inden enhed i klammer, og synes der skal være lidt afstand
\def\myspace{\hspace{5pt}}
% Jeg bruger ofte rod(3) og definerer den her for nemmere adgang
\FProot\RodAfTre{3}{2}%

% Her ændrer jeg i den måde SI pakken viser tal på. Bla ved komma i stedet for punktum
\sisetup{output-decimal-marker = {,}, output-complex-root=\ensuremath{\mathrm{j}}, complex-root-position=before-number, exponent-product = \cdot, output-product = \cdot, detect-weight=true, detect-family=true, detect-shape=true, detect-all=true, per-mode=fraction}

\begin{document}

\newenvironment{Rect=>Polar}[2]%{R}{X}%
{
\StrSubstitute{#1}{,}{.}[\R]%
\StrSubstitute{#2}{,}{.}[\X]%

% Find RX forholdet
%\FPeval\RX{clip(round(\R / \X:2))}%
\FPeval\RX{\R / \X}%
\FPeval\XR{\X / \R}%

% Tag invers
\FParccot\InverseCot{\RX}
%Inversecot: \InverseCot\\

% Fra radianer til grader
\FPeval\AngleResult{\InverseCot*180/\FPpi}%

% cos phi
\FPcos\cosphi{\InverseCot}%

% Find Z
\FPeval\Z{clip(round(\R/\cosphi:2))}%

% Afrund grader
\FPeval\AngleResult{clip(round(\AngleResult:2))}%

\num{\Z}$\angle$\num{\AngleResult}%
}


%%%%%%%%%%%%%%%%%%%%%%%%
\newenvironment{IK,3F=>IK,2F}[3]%{max}{IK3F}{A/kA}
{
\StrSubstitute{#2}{,}{.}[\IK]%

% Regn IK
\FPeval\res{clip(round(\IK * (\RodAfTre / 2):2))}%

\begin{equation}
    \begin{split}
        I_{k2,#1} &= I_{k3,#1} \cdot \frac{\sqrt{3}}{2} \myspace [\si{\ampere}]\\
        &= \num{\IK} \cdot \frac{\sqrt{3}}{2}\\
        &= \SI{\res}{#3}
    \end{split}
\end{equation}
}{}

%%%%%%%%%%%%%%%%%%%%%%%%

\newenvironment{IK,2F=>IK,3F}[3]%{navn}{IK2F}{A/kA}
{
\StrSubstitute{#2}{,}{.}[\IK]%

% Regn IK
\FPeval\res{clip(round(\IK * (2 / \RodAfTre):2))}%

\begin{equation}
    \begin{split}
        I_{k3,#1} &= I_{k2,#1} \cdot \frac{2}{\sqrt{3}} \myspace [\si{\ampere}]\\
        &= \num{\IK} \cdot \frac{2}{\sqrt{3}}\\
        &= \SI{\res}{#3}
    \end{split}
\end{equation}

}{}

%%%%%%%%%%%%%%%%%%%%%%%%

\newenvironment{HV@Sk=>Z}[8]%{net navn}{Sk navn}{Sk}{eksponent}{R/X}{Un}{c}{c navn}
{
\StrSubstitute{#3}{,}{.}[\Sk]%
\StrSubstitute{#4}{,}{.}[\eksponent]%
\StrSubstitute{#5}{,}{.}[\RX]%
\StrSubstitute{#6}{,}{.}[\Un]%
\StrSubstitute{#7}{,}{.}[\c]%

%Gang Sk med eksponent
\FPeval\FirstResult{\Sk * 10^\eksponent}%

% Del Un i anden potens med Sk
\FPeval\Z{(\Un^2) * \c / \FirstResult}%

% Resultat i radianer
\FParccot\InverseCot{\RX}

% Cos af X/R
\FPcos\CosResult{\InverseCot}

% R
\FPeval\R{clip(\CosResult*\Z)}
\FPeval\Rrounded{clip(round(\CosResult*\Z:2))}

% Sin af X/R
\FPsin\SinResult{\InverseCot}

% X
\FPeval\X{clip(\SinResult*\Z)}
\FPeval\Xrounded{clip(round(\SinResult*\Z:2))}

\def\enhed{}%

\FPifgt{\Un}{500}%
\renewcommand\enhed{\ohm}%
\else
% Lav til milli
\FPeval\R{clip(\R * 1000)}
\FPeval\Rrounded{clip(round(\R:2))}
\FPeval\X{clip(\X * 1000)}
\FPeval\Xrounded{clip(round(\X:2))}
\renewcommand\enhed{\milli\ohm}%
\fi

\begin{equation}
    \begin{split}
        \overline{Z}_{#1} &= \frac{U_n^2 \cdot c_{#8}}{S_{kn,#2}} \angle \text{arccot} \left( \frac{R}{X} \right) \myspace [\Omega]\\
        &= \frac{#6^2 \cdot #7}{#3 \cdot 10^#4} \angle \text{arccot} \left( #5 \right)\\
        &=
        \left \{
        \begin{tabular}{r}
            \begin{Rect=>Polar}{\R}{\X}\end{Rect=>Polar}\ \si{\enhed}\\
            {\Rrounded+j\Xrounded}\unit{\enhed}
        \end{tabular}
        \right.
    \end{split}
\end{equation}
}{}

%%%%%%%%%%%%%%%%%%%%%%%%

\newenvironment{HV@Sk=>Ik}[3]%{navn}{Skn}{Unet}
{
\StrSubstitute{#2}{,}{.}[\Skn]%
\StrSubstitute{#3}{,}{.}[\Unet]%

\FPeval\UnetShort{clip(round(\Unet / 1000:2))}%
\FPpow\potens{10}{6}%

\FPeval\IkTreF{clip(round(((\Skn * \potens)/(\Unet * root(2,3)) / 1000):2))}%

% Find IK,2F ved at gange med 2/rod 3
\FPeval\IkToF{clip(round(\IkTreF * (root(2,3) / 2):2))}%

    \begin{equation}
        \begin{split}
            I_{k3,#1} &= \frac{S_{k,#1}}{\sqrt{3} \cdot U_{n}} \myspace \si{[\ampere]}\\
            &= \frac{\num{\Skn e6}}{\sqrt{3} \cdot \num{\UnetShort e3}}\\
            &= \SI{\IkTreF}{\si{\kilo\ampere}}
        \end{split}
    \end{equation}
    
    \begin{equation}
        \begin{split}
            I_{k2,#1} &= I_{k3,#1} \cdot \frac{\sqrt 3}{2} \myspace \si{[\ampere]}\\
            &= \num{\IkTreF} \cdot \frac{\sqrt 3}{2}\\
            &= \SI{\IkToF}{\si{\kilo\ampere}}
        \end{split}
    \end{equation}
}{}

%%%%%%%%%%%%%%%%%%%%%%%%

\newenvironment{HV@I'K,3F=>IK,3F,sek@trafo}[5]%{navn}{trafoNavn}{IK}{U1}{U2}
{
\StrSubstitute{#3}{,}{.}[\IK]%
\StrSubstitute{#4}{,}{.}[\U]%
\StrSubstitute{#5}{,}{.}[\UU]%

\FPeval\res{clip(round(\IK * (\U / \UU) / 1000:2))}%

\begin{equation}
    \begin{split}
        I_{k3,#1,LV} &= I^{\prime}_{k3,#1,HV} \cdot \frac{U_{1,#2}}{U_{2,#2}} \myspace [\si{\ampere}]\\
        &= \num{\IK} \cdot \frac{\num{\U}}{\num{\UU}}\\
        &= \SI{\res}{\kilo\ampere}
    \end{split}
\end{equation}

\begin{equation}
    I_{k1,LV} \approx I_{k3,LV}
\end{equation}
}

%%%%%%%%%%%%%%%%%%%%%%%%

\newenvironment{HV@I'K,FN=>IK,FN,sek@trafo}[4]%{trafoNavn}{IK}{U1}{U2}
{
\StrSubstitute{#2}{,}{.}[\IK]%
\StrSubstitute{#3}{,}{.}[\U]%
\StrSubstitute{#4}{,}{.}[\UU]%

\FProot\RodAfTre{3}{2}%

\FPeval\res{clip(round(\IK * \RodAfTre * (\U / \UU) / 1000:2))}%

\begin{equation}
    \begin{split}
        I_{k1,LV} &= I^{\prime}_{k1} \cdot \sqrt{3} \cdot \frac{U_{1,#1}}{U_{2,#1}} \myspace [\si{A}]\\
        &= \num{\IK} \cdot \sqrt{3} \cdot \frac{\num{\U}}{\num{\UU}}\\
        &= \SI{\res}{\kilo\ampere}
    \end{split}
\end{equation}

\begin{equation*}
    I_{k1,LV} \approx I_{k3,LV}
\end{equation*}
}

%%%%%%%%%%%%%%%%%%%%%%%%



%%%%%%%%%%%%%%%%%%%%%%%%




\section{Ik3 til Ik2}
\begin{IK,3F=>IK,2F}{navn}{12000}{A}
\end{IK,3F=>IK,2F}

\section{Ik2 til Ik3}
\begin{IK,2F=>IK,3F}{navn}{10392,305}{A}
\end{IK,2F=>IK,3F}

\section{Sk til Z}
\begin{HV@Sk=>Z}{max}{min}{120}{6}{0,3}{10000}{1}{max}%{net navn}{Sk navn}{Sk}{eksponent}{R/X}{Un}{c}{c navn}
\end{HV@Sk=>Z}

\section{Sk til Ik}
\begin{HV@Sk=>Ik}{max}{100}{10000}%{navn}{Skn}{Unet}
\end{HV@Sk=>Ik}

\section{I'K,3F til IK,3F,sek over trafo}
\begin{HV@I'K,3F=>IK,3F,sek@trafo}{max}{T5}{500}{10500}{400}%{navn}{trafoNavn}{IK}{U1}{U2}
\end{HV@I'K,3F=>IK,3F,sek@trafo}

\section{I'K,1F til IK,1F,sek over trafo}
\begin{HV@I'K,FN=>IK,FN,sek@trafo}{T5}{400}{10500}{400}%{trafoNavn}{IK}{U1}{U2}
\end{HV@I'K,FN=>IK,FN,sek@trafo}


\end{document}
