%Udarbejdet af Magnus
% Her definerer jeg min "makro" som laver en udregning for mig
\newenvironment{Iz,min,par}[8]%{I-b}{Kt}{Ks}{Ktm}{Kd}{Kn}{Un}{antal kabler}
{
% I følgende udskifter jeg kommaer med punktum i mine variabler. FP-pakken regner med punktum.
% Hver variabel starter med # og derefter nummeret i rækken.
% Fx er SK værdien variabel nummer 3
\StrSubstitute{#1}{,}{.}[\Ib]%
\StrSubstitute{#2}{,}{.}[\Kt]%
\StrSubstitute{#3}{,}{.}[\Ks]%
\StrSubstitute{#4}{,}{.}[\Ktm]%
\StrSubstitute{#5}{,}{.}[\Kd]%
\StrSubstitute{#6}{,}{.}[\Kn]%
\StrSubstitute{#7}{,}{.}[\Un]%
\StrSubstitute{#8}{,}{.}[\Kab]%

%Gang alle korrektionsfaktorer
\FPeval\KorrektionsF{\Kt * \Ks * \Ktm * \Kd * \Kn}

%Del Belastningen med korrektionsfaktor
\FPeval\Iz{clip(round((\Ib / \Kab) / \KorrektionsF:2))}


% Her laver jeg en if/else hvor jeg kigger på spændingen. Er spændingen over 500 V, Regner vi med Kd (dybden i jorden). Er den under kan det være at vi regner på overharmoniske strømme
\FPifgt{\Un}{500}%
\begin{align*}
        I_{z,min} 	&= \frac{I_{B} / n}{K_{t} \cdot K_{s} \cdot K_{tm} \cdot K_{d}}\\
        			&= \frac{\Ib / \Kab}{\Kt \cdot \Ks \cdot \Ktm \cdot \Kd} \\
        			&= \Iz\\
\end{align*}
\else
\begin{align*}
        I_{z,min} 	&= \frac{I_{B} / n}{K_{t} \cdot K_{s} \cdot K_{tm} \cdot K_{n}} \\
        			&= \frac{\Ib / \Kab}{\Kt \cdot \Ks \cdot \Ktm \cdot \Kn} \\
        			&= \Iz\\
\end{align*}
\fi
}


