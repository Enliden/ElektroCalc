%Udarbejdet af Jakob Dupont

% Her definerer jeg min "makro" som laver en udregning for mig
\newenvironment{FraSkTilZcos}[6]%{net navn}{Sk navn}{Sk}{eksponent}{R/X}{Un}
{
% I følgende udskifter jeg kommaer med punktum i mine variabler. FP-pakken regner med punktum.
% Hver variabel starter med # og derefter nummeret i rækken.
% Fx er SK værdien variabel nummer 3
\StrSubstitute{#3}{,}{.}[\Sk]%
\StrSubstitute{#4}{,}{.}[\eksponent]%
\StrSubstitute{#5}{,}{.}[\cosPhi]%
\StrSubstitute{#6}{,}{.}[\Un]%


%Gang Sk med eksponent
\FPeval\FirstResult{\Sk * 10^\eksponent}%

% Del Un i anden potens med Sk
\FPeval\Z{(\Un^2) / \FirstResult}%

% Resultat i radianer
\FParccos\InverseCos{\cosPhi}

% R
\FPeval\R{clip(\cosPhi*\Z)}
\FPeval\Rrounded{clip(round(\cosPhi*\Z:2))}

% Sin af X/R
\FPsin\SinResult{\InverseCos}

% X
\FPeval\X{clip(\SinResult*\Z)}
\FPeval\Xrounded{clip(round(\SinResult*\Z:2))}

\def\enhed{}%

% Her laver jeg en if/else hvor jeg kigger på spændingen. Er spændingen over 500 V regner jeg i ohm - ellers milliohm. På den måde kan jeg bruge det i både LV og HV
\FPifgt{\Un}{500}%
\renewcommand\enhed{\ohm}%
\else
 %Lav til milli
\FPeval\R{clip(\R * 1000)}
\FPeval\Rrounded{clip(round(\R:2))}
\FPeval\X{clip(\X * 1000)}
\FPeval\Xrounded{clip(round(\X:2))}
\renewcommand\enhed{\milli \ohm}%
\fi

\begin{align*}
        \overline{Z}_{#1} &= \frac{U_n^2 \cdot}{S_{kn,#2}} \angle cos^{-1}(cos\varphi)\\
        &= \frac{#6^2}{#3 \cdot 10^#4} \angle cos^{-1} \left( #5 \right)\\
        &= \Rrounded+j\Xrounded \unit{\enhed}
\end{align*}
}


