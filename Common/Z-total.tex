%Udarbejdet af Magnus Overby
% Her definerer jeg min "makro" som laver en udregning for mig
\newenvironment{Ztotal}[3]
%{ R1 | X1 | R2 | X2 | R3 | X3 | R4 | X4 | R5 | X5 }{Z resultat navn}{Z1 navn | Z2 navn | Z3 navn | Z4 navn | Z5 navn}
{
%bestemmer seperator samt læser lister
\setsepchar{ | }
\readlist\Zv{#1}
\readlist\Nam{#3}
%Her udskiftes komma med punktum
\StrSubstitute{\Zv[1]}{,}{.}[\Ret]%
\StrSubstitute{\Zv[2]}{,}{.}[\Xet]%
\StrSubstitute{\Zv[3]}{,}{.}[\Rto]%
\StrSubstitute{\Zv[4]}{,}{.}[\Xto]%
\StrSubstitute{\Zv[5]}{,}{.}[\Rtre]%
\StrSubstitute{\Zv[6]}{,}{.}[\Xtre]%
\StrSubstitute{\Zv[7]}{,}{.}[\Rfi]%
\StrSubstitute{\Zv[8]}{,}{.}[\Xfi]%
\StrSubstitute{\Zv[9]}{,}{.}[\Rfe]%
\StrSubstitute{\Zv[10]}{,}{.}[\Xfe]%
%Da kvadratrod 3 bruges ofte defineres den her.
\FPeval\sqTre{root(2,3)}

%Udregn samlede R og X (ganger 1,8 med R for at få max værdi)
\FPeval\Rtot{round(\Ret + \Rto + \Rtre + \Rfi + \Rfe:2)}
\FPeval\Xtot{round(\Xet + \Xto + \Xtre + \Xfi + \Xfe:2)}

%udregn samlede impedans
\FPeval\Ztot{root(2,pow(2,\Rtot)+pow(2,\Xtot))}

%tests "flag" til hvor mange impedanser der er.
\FPeval\yepET{1}
\FPeval\yepTO{1}
\FPeval\yepTRE{1}
\FPeval\yepFI{1}
\FPeval\yepFE{1}

\FPeval\Zettest{\Ret + \Xet}
\FPifzero{\Zettest}
\begin{centering}
	Ingen impedans, prøv igen\\
\end{centering}
\else
\FPeval\yepET{0}
\fi

\FPeval\Ztotest{\Rto + \Xto + \yepET}
\FPifzero{\Ztotest}
\begin{centering}
	En impedans... hvad regnede du med?\\
\end{centering}
\else
\FPeval\yepTO{0}
\fi

\FPeval\Ztretest{\Rtre + \Xtre + \yepET + \yepTO}
\FPifzero{\Ztretest}
\begin{align*}
	Z_{#2} 	&= Z_{\Nam[1]} + Z_{\Nam[2]}\\
			&= (\Ret + j \Xet) + (\Rto + j \Xto)\\
			&=\Rtot + j\Xtot \Omega	
\end{align*}
\else
\FPeval\yepTRE{0}
\fi

\FPeval\Ztretest{\Rfi + \Xfi + \yepET + \yepTO + \yepTRE}
\FPifzero{\Ztretest}
\begin{align*}
	Z_{#2} 	&= Z_{\Nam[1]} + Z_{\Nam[2]} + Z_{\Nam[3]}\\
			&= (\Ret + j \Xet) + (\Rto + j \Xto) + (\Rtre + j \Xtre)\\
			&=\Rtot + j\Xtot \Omega 	
\end{align*}
\else
\FPeval\yepFI{0}
\fi

\FPeval\Ztretest{\Rfe + \Xfe + \yepET + \yepTO + \yepTRE + \yepFI}
\FPifzero{\Ztretest}
\begin{align*}
	Z_{#2} 	&= Z_{\Nam[1]} + Z_{\Nam[2]} + Z_{\Nam[3]} + Z_{\Nam[4]}\\
			&= (\Ret + j \Xet) + (\Rto + j \Xto) + (\Rtre + j \Xtre) + (\Rfi + j \Xfi)\\
			&=\Rtot + j\Xtot \Omega 	
\end{align*}
\else
\FPeval\yepFE{0}
\fi

\FPeval\all{\yepET + \yepTO + \yepTRE + \yepFI + \yepFE}
\FPifzero{\all}
\begin{align*}
	Z_{#2} 	&= Z_{\Nam[1]} + Z_{\Nam[2]} + Z_{\Nam[3]} + Z_{\Nam[4]} + Z_{\Nam[5]}\\
			&= (\Ret + j \Xet) + (\Rto + j \Xto) + (\Rtre + j \Xtre) + (\Rfi + j \Xfi) + (\Rfe + j \Xfe)\\
			&=\Rtot + j\Xtot \Omega 
\end{align*}
\else
\fi
}


