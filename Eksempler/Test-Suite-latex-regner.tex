\documentclass[a4paper,oneside,10pt,danish]{report}
\usepackage{import}
\subimport{../}{main.tex}
%%Packages
\usepackage[utf8]{inputenc}%https://ctan.org/pkg/inputenc (Accept different input encodings)
\usepackage{amsmath}%https://ctan.org/pkg/amsmath (AMS mathematical facilities for LATEX)
\usepackage{mathtools}%https://ctan.org/pkg/mathtools (Mathematical tools to use with amsmath)
\usepackage{icomma}%https://ctan.org/pkg/icomma (Intelligent commas for decimal numbers)
\usepackage{siunitx}%https://ctan.org/pkg/siunitx (A comprehensive (SI) units package)
\usepackage{fp}% https://ctan.org/pkg/fp?lang=en (Fixed point arithmetic)
\usepackage{xstring}%https://ctan.org/pkg/xstring?lang=en (String manipulation for (LA)TEX)
\usepackage{listofitems}
\usepackage{ifthen}
\usepackage{gensymb} %bl.a. til \degree kommando
%\usepackage{import} % bruges til subimport. Kan godt give problemer og skal måske placeres ude i main.tex

\subimport{Formler/Common/}{fra-Sk-til-Z.tex}
\subimport{Formler/Common/}{fra-Sk-til-Z-cos.tex}
\subimport{Formler/Common/}{fra-ik-til-Z-cos.tex}
\subimport{Formler/Common}{Iz-min-kabeldim.tex}
\subimport{Formler/Common/}{Iz-min-kabeldim-par.tex}
\subimport{Formler/Common/}{Z-trafo.tex}
\subimport{Formler/Common/}{Z-kabel.tex}
\subimport{Formler/Common/}{TrafoFuldlast.tex}
\subimport{Formler/Common/}{Z-total.tex}

\subimport{Formler/Forsyning/}{Z-til-ik-HV.tex}
\subimport{Formler/Forsyning/}{HV-ZtilIk2f.tex}
\subimport{Formler/Forsyning/}{HV-ZtilIk3f.tex}
\subimport{Formler/Forsyning/}{HV-ZtilIk1f-prim.tex}
\subimport{Formler/Forsyning/}{S-jordskinne-trafo.tex}
\subimport{Formler/Forsyning/}{HV-spændingsfald.tex}
\subimport{Formler/Forsyning/}{HV-KBtid-leder.tex}
\subimport{Formler/Forsyning/}{HV-KBtid-skærm.tex}
\subimport{Formler/Forsyning/}{HV-lækstrøm-pr-Fase.tex}

\subimport{Formler/Installation/}{Fasekompensering_med_result_I.tex}
\subimport{Formler/Installation/}{LV-ZtilIk1f-min.tex}
\subimport{Formler/Installation/}{LV-Z-total-max.tex}
\subimport{Formler/Installation/}{LV-Ik1f.tex}
\subimport{Formler/Installation/}{LV-Ik2f.tex}
\subimport{Formler/Installation/}{LV-Ik2f-par-sikr.tex}
\subimport{Formler/Installation/}{LV-Ik3f.tex}
\subimport{Formler/Installation/}{LV-spændingsfald.tex}
\subimport{Formler/Installation/}{Z-kabel-max.tex}


\begin{document}
Her kan man teste sine formler

 
 Her laver vi en kortslutning på parallelle sikringssæt
	\begin{LV-Ik2f,parSikr-kA}
		{ 1 | 2 |
		 10 | 20 }
		{7}
		{2f,min | 
		 kn | 
		 w1}
		%
		%{ R net + trafo | 
			%  X net + trafo | 
			%  R kabler | X kabler}
		% {antal stikledninger}
		% {Ik,navn | net-trafo-navn | kabel-navn}
		%impedanser i milliohm
	\end{LV-Ik2f,parSikr-kA} 
 
 
 
 
 
 
 
 
 
 
 
 
 
 
 
 
 
 
 
 
Fra Sk til Z
\begin{FraSkTilZ}{min}{max}{50}{6}{0,58}{400}
\end{FraSkTilZ}

Kabel dimensionering strøm
%{Ib}{Kt}{Ks}{Ktm}{Kd}{Kn}{Un} (spænding bruges tal at bestemme om det er Kd eller Kn)
\begin{Iz,min}{132}{1,03}{0,85}{0,85}{0,98}{1}{400}
\end{Iz,min}

%{Un}{Sn}{ek}{Pcu}{prim/sek}
\begin{Ztrafo}{10500}{315000}{0,04}{3900}{prim,t3.1}
\end{Ztrafo}

%{L(km)}{r}{x}{Un}{kabelNavn}
\begin{Zkabel}{2,3}{0,206}{0,081}{10000}{Wtot}
\end{Zkabel}

%\begin{Stringtest}{1fx,1f.max.T1,net.min,trafo.T1,W1,6,7,8,1fx,10,11,12,13,14}{3}
%\end{Stringtest}

\begin{ZtilIkHV}{0,5}{0,2}{0,206}{0,075}{0,175}{0,88}{10000}{3fx,1f.max.T1,net.min,trafo.T1,W1}
%{Rnet}{Xnet}{Rkab}{Xkab}{Rtra}{Xtra}{Un}{IkType,IkNavn,NetNavn,TrafoNavn,kabelNavn}
\end{ZtilIkHV}


\begin{TrafoFuldlast}{315000}{10500}{T3.1,prim}
%{Sn}{U_trafo}{trafoNavn,prim/sek}
\end{TrafoFuldlast}

\begin{HV-ZtilIk2f}{10000}{ 0,45 | 2,67 | 0,47 | 0,19 | 0 | 0 | 0 | 0 | 0 | 0 }{Ik2f,T0,min | NT,max | kabel,Wtot | Z3 navn | Z4 navn | Z5 navn}
%{Un}{ R1 | X1 | R2 | X2 | R3 | X3 | R4 | X4 | R5 | X5 }{Ik,navn | Z1 navn | Z2 navn | Z3 navn | Z4 navn | Z5 navn}
\end{HV-ZtilIk2f}

\begin{Ztotal}{ 0,2 | 1,88 | 4.5 | 1,6 | 0 | 0 | 0 | 0 | 0 | 0 }{NT}{Net | Trafo | Z3-navn | Z4-navn | Z5-navn}{1 | 1 | 1 | 1 | 1}
	%{ R1 | X1 | R2 | X2 | R3 | X3 | R4 | X4 | R5 | X5 }{Z resultat navn}{Z1 navn | Z2 navn | Z3 navn | Z4 navn | Z5 navn}{Antal_kab 1 | Antal kabler 2 | Antal kabler 3 | Antal kabler 4 | Antal kabler 5}
\end{Ztotal}

\begin{LV-Ztotal-max}{ 0,88 | 1,34 | 2,45 | 2.1 | 0 | 0 | 0 | 0 | 0 | 0 }{tot,kabel,max}{W1 | W2 | Z3-navn | Z4-navn | Z5-navn}{2 | 1 | 1 | 1 | 1}
	%{ R1 | X1 | R2 | X2 | R3 | X3 | R4 | X4 | R5 | X5 }{Z resultat navn}{Z1 navn | Z2 navn | Z3 navn | Z4 navn | Z5 navn}{Antal_kab 1 | Antal kabler 2 | Antal kabler 3 | Antal kabler 4 | Antal kabler 5}
\end{LV-Ztotal-max}

\begin{LV-Ik1f-kA}{ 4,55 | 3,48 | 5,20 | 2,77 }{1f,min | NT | tot,kabel}
	%{ R net + trafo | X net + trafo | R kabler | X kabler }{Ik,navn | net-trafo-navn | kabel-navn}
	%impedanser i milliohm
\end{LV-Ik1f-kA}

\begin{LV-Ik2f-kA}{ 4,55 | 3,48 | 5,20 | 2,77 }{2f,min | NT | tot,kabel}
	%{ R net + trafo | X net + trafo | R kabler | X kabler }{Ik,navn | net-trafo-navn | kabel-navn}
	%impedanser i milliohm
\end{LV-Ik2f-kA}

\begin{LV-Ik2f,parSikr-kA}{ 4,55 | 3,48 | 5,20 | 2,77 }{3}{2f,par,min | NT | tot,kabel}
	%{ R net + trafo | X net + trafo | R kabler | X kabler }{Ik,navn | net-trafo-navn | kabel-navn}
	%impedanser i milliohm
\end{LV-Ik2f,parSikr-kA}

\begin{LV-Ik3f-kA}{ 4,55 | 3,48 | 5,20 | 2,77 }{3f,min | NT | tot,kabel}
	%{ R net + trafo | X net + trafo | R kabler | X kabler }{Ik,navn | net-trafo-navn | kabel-navn}
	%impedanser i milliohm. Min/max efter impedanser
\end{LV-Ik3f-kA}

\begin{faseKOMP-Iny}{540,8}{0,81}{0,9}{10}{A1}
%{I_belastning}{cosphi_før}{cosphi_ønsket}{Q_batteristørrelse[kVAR]}{Tavlenavn}
\end{faseKOMP-Iny}

\begin{HV-deltaUnet}{130}{0,206}{0,081}{0,9}{W1 | T2}
	%{Ib}{R}{X}{cosPhi}{U navn | Ib navn}
\end{HV-deltaUnet}

\end{document}
